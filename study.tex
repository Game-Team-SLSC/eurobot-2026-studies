\documentclass{article}
\usepackage{graphicx}
\usepackage{amsmath}
\usepackage{frenchmath}
\usepackage{hyperref}


\title{Robot}
\author{Maxime Caron}
\date{October 2025}

\begin{document}

\maketitle

\section{Introduction}

Ce document présente les calculs mécaniques du robot Game Team. Toutes les valeurs sont traitées selon les règles de chiffres significatifs et le Système International.

\section{Bras}

\subsection{Emetteur ventouse}

La masse des élèments du L3 incluant la palette hors servo = 168g d'après Fusion 360.
On ne considère que le poids comme force agissant sur le bras.
$$P = m \cdot g = 0.168 \cdot 9.81 = \boxed{1.65\,\mathrm{N}}$$
Calculons à présent le couple exercé par cette force sur le bras.
Soit la distance entre le centre de gravité de l'émetteur et l'axe de rotation du bras :
$$d = 5.0\,\mathrm{cm} = 0.050\,\mathrm{m}$$
Le couple est donné par :
$$C = P \cdot d = 1.65 \cdot 0.050 = \boxed{8.3 \times 10^{-2}\,\mathrm{N\cdot m}} = \boxed{8.5 \times 10^{-1}\,\mathrm{kgf\cdot cm}}$$
Le servo choisi doit être capable de fournir au moins ce couple multiplié par 3 (facteur de sécurité pour couvrir dynamiques, frottements et chocs) :
$$C_{\min} = 3 \times 8.3\times10^{-2} = \boxed{2.49\times10^{-1}\,\mathrm{N\cdot m}} \approx \boxed{2.55\,\mathrm{kgf\cdot cm}}$$

\subsection{Lien 2}

La masse des élèments de l'émetteur + J1 incluant la palette hors servo = 200g d'après Fusion 360.
On ne considère que le poids comme force agissant sur le bras.
$$P = m \cdot g = 0.200 \cdot 9.81 = \boxed{1.96\,\mathrm{N}}$$
Calculons à présent le couple exercé par cette force sur le bras.
Soit la distance entre le centre de gravité de l'émetteur et l'axe de rotation du bras :
$$d = 150.0\,\mathrm{mm} = 0.150\,\mathrm{m}$$
Le couple est donné par :
$$C = P \cdot d = 1.96 \cdot 0.150 = \boxed{2.94 \times 10^{-1}\,\mathrm{N\cdot m}} = \boxed{3.0\,\mathrm{kgf\cdot cm}}$$
Le servo choisi doit être capable de fournir au moins ce couple multiplié par 3 (facteur de sécurité pour couvrir dynamiques, frottements et chocs) :
$$C_{\min} = 3 \times 2.94\times10^{-1} = \boxed{8.82\times10^{-1}\,\mathrm{N\cdot m}} \approx \boxed{8.99\,\mathrm{kgf\cdot cm}}$$

\subsection{Lien 1}
La masse des élèments de l'émetteur + J1 + J2 incluant la palette hors servo = 250g d'après Fusion 360.
On ne considère que le poids comme force agissant sur le bras.
$$P = m \cdot g = 0.250 \cdot 9.81 = \boxed{2.45\,\mathrm{N}}$$
Calculons à présent le couple exercé par cette force sur le bras.
Soit la distance entre le centre de gravité de l'émetteur et l'axe de rotation du bras :
$$d = 180.0\,\mathrm{mm} = 0.180\,\mathrm{m}$$
Le couple est donné par :
$$C = P \cdot d = 2.45 \cdot 0.180 = \boxed{4.41 \times 10^{-1}\,\mathrm{N\cdot m}} = \boxed{4.5\,\mathrm{kgf\cdot cm}}$$
Le servo choisi doit être capable de fournir au moins ce couple multiplié par 3 (facteur de sécurité pour couvrir dynamiques, frottements et chocs) :
$$C_{\min} = 3 \times 4.41\times10^{-1} = \boxed{1.32\,\mathrm{N\cdot m}} \approx \boxed{13.5\,\mathrm{kgf\cdot cm}}$$

\subsection{Aspiration}
Les forces exercées sur la palette sont (en négligeant le frottement et la rotation) :
$$\sum F = F_{poids} + F_{inertie}$$
Le poids de la palette vaut :
$$F_{poids} = m \cdot g = 0.150 \cdot 9.81 = \boxed{1.47\,\mathrm{N}}$$
La pompe à vide fournit une dépression (pression relative) de :
$$\Delta p_{pompe} = 420\,\mathrm{mmHg} \approx 0.550\,\mathrm{bar} = \boxed{55\,\mathrm{kPa}} = \boxed{5.50 \times 10^{4}\,\mathrm{Pa}}$$

L'inertie (accélération faible) :
$$F_{inertie} = m \cdot a = 0.150 \times 0.01 = 1.5 \times 10^{-3}\,\mathrm{N}$$

Donc la force totale exercée sur la palette est :
$$F = F_{poids} + F_{inertie} = 1.47 + 1.5 \times 10^{-3} \approx \boxed{1.47\,\mathrm{N}}$$

Pour la ventouse :
$$F = \Delta p \cdot S \;\;\Leftrightarrow\;\; S = \frac{F}{\Delta p}$$
$$S = \frac{1.47}{5.50 \times 10^{4}} = 2.67 \times 10^{-5}\,\mathrm{m^2} = \boxed{26.7\,\mathrm{mm^2}}$$

La ventouse a une surface circulaire, donc
$$r = \sqrt{\frac{S}{\pi}} = \sqrt{\frac{26.7}{\pi}} = \boxed{2.92\,\mathrm{mm}}$$
Avec facteur de sécurité de 4 on obtient :
$$r_{final} = 4 \times 2.92 = \boxed{11.7\,\mathrm{mm}}$$
Or il n'existe pas de ventouse de ce rayon, on prendra donc une ventouse de rayon 20mm (qui permet d'assurer l'étanchéité).
Il la faut en silicone et de course 1cm environ, souple (40-60 Shore A).
\section{Convoyeur}
\subsection{Dimensions}
La largeur de chaque palette verticalement entre les palettes est de $W = 30\,\mathrm{mm}$.
Pour que chaque pic ait à la fois solidité et souplesse, on estime que l'épaisseur doit être de 3mm
$$e = 3\,\mathrm{mm}$$
Il faut également prendre en compte l'épaisseur du tapis roulant :
$$t = 1.5\,\mathrm{mm}$$
La circonférence du demi cercle de l'axe est :
$$C_{poulie} = \pi \cdot r$$
Il doit être possible pour le convoyeur d'avoir (en bout de course), un pic en haut et un pic en bas. Le diamètre de l'axe doit donc être :
$$W + e = C_{axe} + t \Leftrightarrow r = \frac{W + e - t}{\pi} = 10.03\, \mathrm{mm} \approx 10\,\mathrm{mm}$$

donc le diamètre de l'axe est :
$$d_{axe} = 2r = \boxed{20\,\mathrm{mm}}$$

\subsection{Poulie entrainée}
Sachant que la poulie driven a un diamètre extérieur de 20mm, déterminons le nombre de pas pour celle ci en format T5 :
$$N_{dents}  = \frac{\pi \cdot d}{p} = \frac{\pi \cdot 20}{5} = 12.57 \approx 13\,\text{dents}$$

\subsection{Poulie motrice}
Dimensionnement arbitraire du moteurpour l'instant :
$$C = 0.1\,\mathrm{N\cdot m}$$

\section{Propulsion}
\subsection{Dimensionnement des moteurs}
Le robot pèsera environ $m = 4\,\mathrm{kg}$. \\
Il sera entrainé par 4 moteurs qui seront liées à des roues meccanum de diamètre $d_{roue} = 60\,\mathrm{mm}$. \\
Il devra atteindre 2 m/s en vitesse de pointe avec une accélération de $a = 1.5\,\mathrm{m/s^2}$. \\
Calculons le rendements :
- En passant à 16V
- Rendement standart d'un stepper : 50\%
- Perte mécanique (roues meccanum) : 25\%
$$\eta = 0.50 \times 0.75 = 0.38 = 38\%$$

D'après \href{https://community.robotshop.com/blog/show/dimensionnement-dun-moteur-dentranement}{voir le site pour le détail du calcul}

$$C = 0.23\,\mathrm{N\cdot m}$$

En ajoutant un facteur de sécurité de 2 pour couvrir les pertes dynamiques, frottements et chocs, on obtient :
$$C_{final} = 2 \times 0.23 = \boxed{0.46\,\mathrm{N\cdot m}}$$

\newpage
\section{Pince - ARCHIVE}
\subsection{Ecart entre palettes}
Soit chaque palette sur le plan latéral d’un cube de $5.0\,\mathrm{cm} \times 3.0\,\mathrm{cm}$ :

$$l_{palette} = 5.0\,\mathrm{cm} \qquad h_{palette} = 3.0\,\mathrm{cm}$$

En position repliée, l'écart entre les bras vaut :
$$E_{plie} = l_{palette} = 5.0\,\mathrm{cm}$$
Pour la rotation libre, l'écart (déployé) doit être égal à deux fois la projection de la palette sur l’axe horizontal :

$$E_{theo} = 2T = 2 \cdot \frac{\sqrt{l_{palette}^2 + h_{palette}^2}}{2} = \sqrt{5.0^2 + 3.0^2} = 5.8\,\mathrm{cm}$$

Un jeu $J = 0.3\,\mathrm{cm}$ (3 mm) est ajouté pour compenser les imprecisions :

$$E_{final} = E_{theo} + J = 5.8 + 0.3 = \boxed{6.1\,\mathrm{cm}}$$

\subsection{Positionnement des bielles sur la manivelle}

On traite un côté (à reproduire symétriquement) :

$$d_{O/A,plie} = \frac{E_{plie}}{2} = 2.5\,\mathrm{cm}$$
$$d_{O/A,deploye} = \frac{E_{final}}{2} = 3.05\,\mathrm{cm}$$
$$d_{O/B,plie} = d_{O/A,plie} + E_{plie} = 2.5 + 5.0 = 7.5\,\mathrm{cm}$$
$$d_{O/B,deploye} = d_{O/A,deploye} + E_{final} = 3.05 + 6.1 = 9.15\,\mathrm{cm}$$

Déplacement de la tête de bielle :
$$\Delta_{O/A} = d_{O/A,deploye} - d_{O/A,plie} = 3.05 - 2.5 = 0.55\,\mathrm{cm}$$
$$\Delta_{O/B} = d_{O/B,deploye} - d_{O/B,plie} = 9.15 - 7.5 = 1.65\,\mathrm{cm}$$

La course de la manivelle vaut $d = 2r$, donc $r = d/2$ :
$$r_{proche} = \frac{\Delta_{O/A}}{2} = 0.275\,\mathrm{cm} = \boxed{0.28\,\mathrm{cm}}$$
$$r_{eloigne} = \frac{\Delta_{O/B}}{2} = 0.825\,\mathrm{cm} = \boxed{0.83\,\mathrm{cm}}$$

\includegraphics[scale=0.5]{res/mechanism}

\subsection{Calcul du couple inertiel}

Le bras est guidé linéairement : on néglige gravité et frottement. Seule l’inertie est modélisée :

$$F_i = m \cdot a = m \cdot \omega^2 \cdot r$$
$$C = F_i \cdot r = m \cdot \omega^2 \cdot r^2$$

Données :
\begin{itemize}
    \item Masse piston + palette : $m = 0.190\,\mathrm{kg}$
    \item Vitesse manivelle : $\omega = \pi\,\mathrm{rad\,s}^{-1}$
    \item Rayons : $r_{proche} = 0.0028\,\mathrm{m}$ ; $r_{eloigne} = 0.0083\,\mathrm{m}$
\end{itemize}

Calculs :
$$C_{proche} = 0.190 \times \pi^2 \times (0.0028)^2 \approx 1.47 \times 10^{-5}\,\mathrm{N\cdot m}$$
$$C_{eloigne} = 0.190 \times \pi^2 \times (0.0083)^2 \approx 1.29 \times 10^{-4}\,\mathrm{N\cdot m}$$

Pour 2 bielles " proches " et 2 " éloignées " : 
$$C_{total} = 2\,C_{proche} + 2\,C_{eloigne} \approx 2.87 \times 10^{-4}\,\mathrm{N\cdot m}$$

Ajouter un facteur de sécurité (x3 à x4) pour le choix du servo.

\subsection{Matériaux (impression 3D)}

La densité effective :
$$\rho_{impression} = \rho_{materiau} \cdot \frac{Infill}{100} \cdot k$$
où $k$ dépend du motif d’infill :

\vspace{0.5em}
\begin{tabular}{|c|c|}
    \hline
    Infill & Coefficient $k$ \\
    \hline
    100\%   & 0.92 à 0.98 (typiquement 0.95) \\
    80\%    & 0.90 à 0.95 \\
    60\%    & 0.85 à 0.92 \\
    40\%    & 0.80 à 0.88 \\
    20\%    & 0.75 à 0.85 \\
    \hline
\end{tabular}

\vspace{1em}
\begin{tabular}{|l|c|}
    \hline
    Matériau & $\rho \, (\mathrm{g\,mm}^{-3})$ \\
    \hline
    Polycarbonate (PC) & $1.21\times10^{-3}$ \\
    PLA                & $1.24\times10^{-3}$ \\
    \hline
\end{tabular}

Exemples (infill 60\%, $k=0.85$) :
$$\rho_{bras} = 1.21\times10^{-3} \times 0.6 \times 0.85 = \boxed{6.17\times10^{-4}\,\text{g\,mm}^{-3}}$$
$$\rho_{bielle} = 1.24\times10^{-3} \times 0.6 \times 0.85 = \boxed{6.32\times10^{-4}\,\text{g\,mm}^{-3}}$$

\end{document}
